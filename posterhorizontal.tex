
\documentclass[final]{beamer}



\usepackage[scale=0.8,size=a1]{beamerposter} % Use the beamerposter package for laying out the poster

\usetheme{confposter} % Use the confposter theme supplied with this template
\usepackage{multicol}
\setbeamercolor{block title}{fg=dblue,bg=white} % Colors of the block titles
\setbeamercolor{block body}{fg=black,bg=white} % Colors of the body of blocks
\setbeamercolor{block alerted title}{fg=white,bg=dblue!70} % Colors of the highlighted block titles
\setbeamercolor{block alerted body}{fg=black,bg=dblue!10} % Colors of the body of highlighted blocks
% Many more colors are available for use in beamerthemeconfposter.sty

%-----------------------------------------------------------
% Define the column widths and overall poster size
% To set effective sepwid, onecolwid and twocolwid values, first choose how many columns you want and how much separation you want between columns
% In this template, the separation width chosen is 0.024 of the paper width and a 4-column layout
% onecolwid should therefore be (1-(# of columns+1)*sepwid)/# of columns e.g. (1-(4+1)*0.024)/4 = 0.22
% Set twocolwid to be (2*onecolwid)+sepwid = 0.464
% Set threecolwid to be (3*onecolwid)+2*sepwid = 0.708

\newlength{\sepwid}
\newlength{\onecolwid}
\newlength{\twocolwid}
\newlength{\threecolwid}
\setlength{\paperwidth}{33.1in} % A0 width: 46.8in
\setlength{\paperheight}{23.4in} % A0 height: 33.1in
\setlength{\sepwid}{0.0\paperwidth} % Separation width (white space) between columns
\setlength{\onecolwid}{0.22\paperwidth} % Width of one column
\setlength{\twocolwid}{0.464\paperwidth} % Width of two columns
\setlength{\threecolwid}{0.708\paperwidth} % Width of three columns
\setlength{\topmargin}{-0.5in} % Reduce the top margin size
%-----------------------------------------------------------

\usepackage{graphicx}  % Required for including images

\usepackage{booktabs} % Top and bottom rules for tables

%----------------------------------------------------------------------------------------
%	TITLE SECTION 
%----------------------------------------------------------------------------------------

\title{Exploring the dynamic of changes: A model to understand the amphorae production patterns in the Roman Empire} % Poster title

\author{Gabacho, Mariquilla and el rubio} % Author(s)

\institute{Barcelona Supercomputing Center - University of Barcelona} % Institution(s)

%----------------------------------------------------------------------------------------

\begin{document}

\addtobeamertemplate{block end}{}{\vspace*{2ex}} % White space under blocks
\addtobeamertemplate{block alerted end}{}{\vspace*{2ex}} % White space under highlighted (alert) blocks

\setlength{\belowcaptionskip}{2ex} % White space under figures
\setlength\belowdisplayshortskip{2ex} % White space under equations

\begin{frame}[t] % The whole poster is enclosed in one beamer frame

\begin{columns}[t] % The whole poster consists of three major columns, the second of which is split into two columns twice - the [t] option aligns each column's content to the top

\begin{column}{\sepwid}\end{column} % Empty spacer column

\begin{column}{\onecolwid} % The first column


\begin{block}{Introduction}

Material culture variability allows to understand the interpretation of the change processes, focused on the production of olive oil amphorae (called Dressel 20) during the Roman Empire. \\
In particular, we want to identify changes on the pattern productions and if these changes were produced by economical and political reasons ~\cite{schillinger}. As
hypothesis, we think that making techniques processes were transmitted by vertical transmission based on the learning production techniques from master to disciple instead of horizontal transmission where this learning is done by workers with the same level. If this hypothesis is validated amphorae made in nearby workshops should share more traits than amphorae made from farthest workshops. Empirical studies and theoretical model were propose to test this hypothesis. 
\end{block}


\begin{block}{Dataset}


We analysed a sample of 413 amphora shapes from 4 different workshops showed in the map \textbf{(fig.1)}.These workshops were selected from different sites of \emph{Baetica} province in order to know if morphometric distance was correlated with spatial distance. A set 90 samples from each workshop were used to create a database. 8 measures focused on the rim were taken from each amphorae sample. 

\begin{figure}
\includegraphics[width=0.7\linewidth]{images/fig1.png}
\caption{Pottery workshops were distributed along rivers Guadalquivir and Genil. Red circles belong to the workshops analysed}
\end{figure}


 \end{block}
\end{column} % End of the first column

%BEGIN THE SECOND COLUMN-------------------------------------------------
\begin{column}{\twocolwid}


\begin{block}{Multivariate analysis}
\begin{columns}[t,totalwidth=\twocolwid]



\begin{column}{\onecolwid} %first subcolumn left


{\textbf{Principal Component Analysis}} 
\justify

PCA was used to explore these metrical observations with the 8 measurement as variables. Results allow us to simplify the analysis by grouping the variance of the dataset. The first two principal components were chosen to see the significant differences among workshops. 

\vspace{1cm}
{\textbf{Results}}\\
\justify
\textbf{Figure 2} shows the workshops with a minor space such Bel\'en and Malpica share more pottery traits than the rest: Parlamento and Las Delicias.


\end{column}

\begin{column}{\sepwid}\end{column} % Empty spacer column

\begin{column}{\onecolwid} %first subcolumn right


\begin{figure}
\includegraphics[width=0.6\linewidth]{images/fig2.png}
\caption{First and Second Principal Components for the analysed sample}
\end{figure}

\end{column}
 % End of the second column
\end{columns}

\end{block}

\begin{block}{Theoretical Exploration}

\begin{columns}[t,totalwidth=\twocolwid]

\begin{column}{\onecolwid} %first subcolumn left
%on the left

{\textbf{Model}}\\
\justify

We propose a simple model where a fixed number of workshop are producing a fixed amount of amphora during a certain time. 
To simplify the problem the amphora production is describe by only one measure among the measure previously observed (external diameter) and each workshop produce amphora where this measure follow a normal distribution. The parameter of this normal distribution (the mean value and the standard deviation) are the cultural knowledge that workshop can or cannot exchange given the different setup. Every 100 time step each workshop have a certain probability to modify those parameter and/or to copy the one used by another workshop. 


\end{column}

\begin{column}{\sepwid}\end{column} % Empty spacer column

\begin{column}{\onecolwid} %first subcolumn right
on the right
\end{column}
 % End of the second column
\end{columns}

\end{block}
\end{column}

%BEGIN LAST COLUMN----------------------------------------------------

\begin{column}{\sepwid}\end{column} % Empty spacer column

\begin{column}{\onecolwid} % The third column

\begin{block}{Concluding Remarks}

Empirical studies identified a variation on the making techniques processes among pottery workshops. We observe that this variability is affected by the distance: amphorae made in nearby workshops with a minor spacial distance share more traits than amphorae made in pottery workshops farther. It suggests that the pottery techniques were learned from master to disciple instead of workers with the same level. 
 
\end{block}

\begin{block}{References}
\small

\begin{thebibliography}{50}

\bibitem[1]{mesoudi}\textsc{Mesoudi, A. (2015)}
\textit{Cultural Evolution: A review of Theory, Finding and Controversies}, Evolutionary biology.

\bibitem[2]{agui}\textsc{Aguilera, A. (1998)}
\textit{An\'alisis multivariable: una nueva v\'ia para la caracterizaci\'on cer\'amica}, Pyranae, 29.

\bibitem[3]{schillinger}\textsc{Schillinger, K. et al. (2006)}
\textit{Differences in Manufacturing Traditions and Assemblage-Level Patterns: the Origins of Cultural Differences in Archaeological Data}, Journal of Archaeological Method Theory.

\bibitem[4]{li}\textsc{Li, A. (2014)}
\textit{Crossbows and imperial craft organisation: the bronze triggers of China's Terracotta Army}, Antiquity, 88.339.

\end{thebibliography}

\end{block}

%----------------------------------------------------------------------------------------
%	ACKNOWLEDGEMENTS
%----------------------------------------------------------------------------------------

\setbeamercolor{block title}{fg=dblue,bg=white} % Change the block title color

\begin{block}{Acknowledgements}

\small{\rmfamily{The Funding for this work was provided by the ERC Advanced Grant EPNet (340828)}}

\end{block}

%----------------------------------------------------------------------------------------
%	CONTACT INFORMATION
%----------------------------------------------------------------------------------------

\setbeamercolor{block alerted title}{fg=white,bg=dblue!70} % Change the alert block title colors
\setbeamercolor{block alerted body}{fg=black,bg=white} % Change the alert block body colors


\begin{center}
\begin{tabular}{ccc}
\includegraphics[width=0.4\linewidth]{images/epnet.png} & \hfill & \includegraphics[width=0.4\linewidth]{images/erc.png}
\end{tabular}
\end{center}

%----------------------------------------------------------------------------------------

\end{column} % End of the third column

\end{columns} % End of all the columns in the poster

\end{frame} % End of the enclosing frame

\end{document}
